% Flipping a coin
% Author: cis
\documentclass[border=10pt,varwidth]{standalone}
%%%<
\usepackage{verbatim}
%%%>
\begin{comment}
:Title: Tree of probabilities - flipping a coin
:Tags: Trees;Nodes and shapes;Mathematics
:Author: cis
:Slug: coin-flipping

In mathematics, specifically in statistics, coin-flipping is an introductory
example of the complexities of probability and statistics.

This example was written by cis on TikZ.de.
http://tikz.de/baum/
\end{comment}
\usepackage{tikz}
\usetikzlibrary{calc, shapes, backgrounds}
\usepackage{amsmath, amssymb}
%\pagecolor{olive!50!yellow!50!white}
\begin{document}
\tikzset{
  healthy/.style = {fill = orange!90!blue,
                 label = center:\textsf{\Large H}},
  disease/.style = {fill = blue!70!yellow, text = black,
                 label = center:\textsf{\Large D}},
  hp/.style = {fill = orange!90!blue,
                 label = center:\textsf{H\&P}},
  dp/.style = {fill = blue!70!yellow, text = black,
                 label = center:\textsf{D\&P}},
   hn/.style = {fill = orange!90!blue,
                 label = center:\textsf{H\&N}},
  dn/.style = {fill = blue!70!yellow, text = black,
                 label = center:\textsf{D\&N}}                                
}
\begin{tikzpicture}[
    scale = 1.5, transform shape, thick,
    every node/.style = {draw, circle, minimum size = 10mm},
    grow = down,  % alignment of characters
    level 1/.style = {sibling distance=3.5cm},
    level 2/.style = {sibling distance=2cm}, 
    level 3/.style = {sibling distance=3cm}, 
    level distance = 1.25cm
  ]
%  \node[fill = gray!40, shape = rectangle, rounded corners,
  %  minimum width = 6cm, font = \sffamily] {Coin flipping} 
  \node [shape = circle split, draw, line width = 1pt,
          minimum size = 10mm, inner sep = 0mm, font = \sffamily\large,
          rotate=30] (Start)
          { \rotatebox{-30}{H} \nodepart{lower} \rotatebox{-30}{D}}
   child { node [healthy] (A) {}
     child {node [hn] (B) {}}
     child { node [hp] (C) {}}
   }
   child {  node [disease] (D) {}
     child { node [dp] (E) {}}
     child { node [dn] (F) {}}
   } ;
%  };

  % Filling the root (Start)
  \begin{scope}[on background layer, rotate=30]
    \fill[healthy] (Start.base) ([xshift = 0mm]Start.east) arc (0:180:5mm)
      -- cycle;
   \fill[disease] (Start.base) ([xshift = 0pt]Start.west) arc (180:360:5mm)
      -- cycle;
  \end{scope}

  % Labels
  \begin{scope}[nodes = {draw = none}]
    \path (Start) -- (A) node [near start, left]  {$0.9$};
    \path (A)     -- (B) node [near start, left]  {$0.95$};
    \path (A)     -- (C) node [near start, right] {$0.05$};
    \path (Start) -- (D) node [near start, right] {$0.1$};
    \path (D)     -- (E) node [near start, left]  {$0.98$};
    \path (D)     -- (F) node [near start, right] {$0.02$};
    \begin{scope}[nodes = {below = 11pt}]
      \node [name = X] at (B) {$0.855$};
      \node            at (C) {$0.045$};
      \node [name = Y] at (E) {$0.098$};
      \node            at (F) {$0.002$};
    \end{scope}
%    \draw[densely dashed, rounded corners, thin]
  %    (X.south west) rectangle (Y.north east);
  \end{scope}
\end{tikzpicture}
\end{document}